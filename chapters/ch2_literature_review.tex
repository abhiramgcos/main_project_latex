\chapter{LITERATURE REVIEW}

\section{IoT Firmware Analysis and Emulation}

Research in automated firmware analysis evolved significantly over the past decade as security implications of firmware vulnerabilities became apparent. Early approaches relied primarily on static analysis examining firmware binaries without execution. While static analysis provides valuable insights into code structure, it fundamentally cannot capture runtime behaviors, inter-process communications, or network interactions revealing security flaws. This limitation motivated development of dynamic analysis frameworks based on firmware emulation.

Chen et al.~\cite{chen2016} introduced Firmadyne in 2016 as a pioneering framework for automated large-scale firmware emulation. Firmadyne automatically extracts filesystem contents, identifies embedded kernels, configures QEMU-based virtual machines, sets up network interfaces, and launches emulated instances. The framework demonstrated feasibility of automated firmware analysis at scale, revealing widespread vulnerabilities including backdoor accounts, web interface vulnerabilities, insecure network services, and inadequate input validation. However, Firmadyne achieved only 16.28\% emulation success across diverse collections due to rigid hardware and network configuration assumptions.

Subsequent research at KAIST's SysSec Lab~\cite{kim2020} developed advanced arbitration techniques improving emulation success. Rather than attempting perfect hardware replication, arbitrated emulation prioritizes creating functional environments sufficient for security analysis. Researchers identified five critical arbitration categories: boot process arbitration handling initialization failures, network configuration arbitration establishing connectivity, NVRAM arbitration emulating persistent storage, kernel compatibility arbitration managing version mismatches, and environment arbitration addressing miscellaneous setup issues. Implementation of systematic arbitration achieved 79.36\% emulation success across 1,124 firmware images---nearly five-fold improvement over previous approaches. The arbitrated approach also enabled discovery of 12 new zero-day vulnerabilities affecting 23 devices.

\section{Network Scanning and Device Discovery}

Network scanning encompasses techniques for discovering active hosts, identifying services, and fingerprinting systems. For IoT security, comprehensive network visibility is essential as administrators cannot secure devices they don't know exist. Traditional scanning tools like Nmap~\cite{nmap} provide powerful capabilities but lack IoT-specific features.

The Address Resolution Protocol (ARP) operates at Layer 2 of the OSI model providing mechanism to map IP addresses to hardware MAC addresses. ARP-based scanning offers significant advantages for IoT device discovery. When hosts need to communicate with IP addresses on local networks, they broadcast ARP requests. Network scanners exploit this by sending ARP requests to all subnet addresses and recording responses. ARP scanning bypasses firewall rules typically operating at Layer 3 and above. Devices configured to ignore ICMP pings still respond to ARP requests.

Research demonstrated Python-based ARP scanning implementations using Scapy library for packet crafting. Scapy provides low-level network access enabling construction of custom ARP packets, broadcast sending, response capture and parsing, and IP/MAC address extraction. Implementation considerations include multi-threading for concurrent address scanning, appropriate timeout values balancing speed and reliability, retry mechanisms handling packet loss, and rate limiting avoiding network infrastructure overload.

TCP port scanning reveals services devices are running. Several techniques exist with different tradeoffs. TCP SYN scanning sends SYN packets observing responses without completing handshakes---faster and stealthier than full connections. Service version detection connects to open ports analyzing banners for software identification. Research demonstrated operating system fingerprinting through port pattern analysis as different systems have characteristic open port configurations. For IoT devices specifically, non-standard ports and minimal service footprints present challenges requiring broader port range testing.

\section{Machine Learning for Malware Detection}

Malware detection evolved through several technology generations. First-generation signature-based detection matched known malware signatures but completely failed for novel threats. Second-generation heuristic analysis examined code behavior patterns detecting suspicious activities without specific signatures, though generating high false positives. Third-generation traditional machine learning applying KNN, SVM, Decision Trees, and Random Forests improved over heuristics but required extensive manual feature engineering and struggled with high-dimensional data. Fourth-generation deep learning automatically learns hierarchical feature representations from raw data, eliminating manual engineering while identifying subtle patterns across millions of parameters.

Research into deep learning for malware detection explored various architectures. Convolutional Neural Networks originally developed for image classification excel at identifying spatial patterns. Researchers adapted CNNs to malware detection by representing malware as images with binary code visualized as pixels or network traffic converted to image format. CNNs automatically learn convolutional filters detecting characteristic patterns at multiple scales. Khan et al.~\cite{khan2023} demonstrated deep boosted CNN for IoT malware detection converting network packets into grayscale images, achieving 98.6\% detection accuracy through ensemble approaches.

Recurrent Neural Networks process sequential data maintaining hidden states capturing temporal dependencies. For malware detection, RNNs analyze sequences of system calls, API calls, or network packets. Long Short-Term Memory variants address vanishing gradient problems enabling learning of long-range dependencies. Al Abbas et al.~\cite{abbas2023} applied RNNs with NLP techniques to IoT malware detection treating binary code as text sequences, achieving 99.3\% accuracy using word embeddings for assembly instructions and RNNs capturing execution flow patterns.

Hybrid architectures combin multiple network types leverage complementary strengths. Jeyalakshmi \& Nallaperumal~\cite{jeyalakshmi2024} introduced HCAGAN-DBN combining GANs and DBNs. Their GAN generated synthetic training samples to address class imbalance while the DBN did hierarchical feature learning, achieving 99.83\% accuracy. Cross-architecture malware detection addresses IoT's diverse processor architectures. Chaganti et al.~\cite{chaganti2022} developed architecture-agnostic feature extraction capturing semantic behaviors regardless of instruction set, achieving 100\% malware identification accuracy and 98\% family classification across architectures.

\section{Integrated Security Frameworks}

While significant research addresses individual IoT security components, fewer works examine integrated approaches combining multiple domains. Security Information and Event Management platforms aggregate security data from multiple sources enabling correlation and analysis. Key SIEM principles include data normalization into common schemas, real-time stream processing for continuous analysis, correlation rules identifying patterns across events, and automated response executing predefined playbooks.

Security Orchestration, Automation and Response platforms extend SIEM with enhanced automation. SOAR characteristics include workflow automation defining tasks as directed graphs, integration capabilities providing standardized APIs for diverse tools, playbook libraries for common security scenarios, and human-in-the-loop support for critical decisions requiring approval. Modern distributed systems increasingly adopt containerization and microservices architectures. Container technologies like Docker provide application isolation, consistent environments, efficient resource utilization, and rapid deployment capabilities. Microservices architecture breaking systems into loosely-coupled services provides modularity, independent scalability, resilience, and technology diversity.

Our literature review reveals substantial progress in individual domains while highlighting critical need for integrated solutions. Key findings informed our system design: arbitrated emulation prioritizing functional environments over perfect hardware replication achieves practical success rates; ARP-based scanning provides superior IoT device discovery bypassing firewall restrictions; deep learning with CNNs achieves state-of-the-art accuracy generalizing to novel malware variants; and integrated platforms combining multiple security capabilities provide superior threat visibility compared to fragmented toolsets. By synthesizing insights from these research areas, we developed FIRMAI as comprehensive integrated IoT security platform addressing critical gaps in existing solutions.