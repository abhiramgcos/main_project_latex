\chapter{System Implementation}

\section{Introduction}
This chapter explains the complete implementation of the proposed system FIRMAI: AI-Powered IoT Firmware Vulnerability Analyzer. It describes how the system architecture designed in the previous chapters is translated into a working software solution. The implementation focuses on integrating firmware analysis, network scanning, and AI-based malware detection into a single unified platform. Special emphasis is given to modular design, automation, scalability, and security.

The system is implemented using open-source tools and modern software frameworks to ensure flexibility, reliability, and ease of deployment. Each module is developed independently and later integrated through well-defined interfaces to provide end-to-end IoT security analysis.

\section{Overall System Implementation Architecture}
The FIRMAI system follows a modular and service-oriented architecture, where each functional component operates as an independent module. These modules communicate using RESTful APIs and asynchronous message queues, ensuring loose coupling and high scalability.

The implementation is divided into four logical layers:
\begin{itemize}
	\item \textbf{Presentation Layer} -- Provides web-based dashboard and API interfaces for user interaction.
	\item \textbf{Application Logic Layer} -- Implements firmware analysis, network scanning, vulnerability detection, and malware classification logic.
	\item \textbf{Data Management Layer} -- Handles structured and unstructured data storage.
	\item \textbf{Infrastructure Layer} -- Manages containerization, networking, and deployment services.
\end{itemize}

\section{Firmware Acquisition and Processing Module}
The firmware acquisition module is responsible for collecting IoT firmware images for analysis. Firmware images can be obtained through multiple methods such as manual upload, direct extraction from IoT hardware devices, or interception of Over-The-Air (OTA) update traffic using a transparent proxy.

Once a firmware image is collected, it is processed using Binwalk, which identifies embedded filesystems, compressed archives, and executable binaries. The extraction process recursively unpacks nested components and reconstructs the complete filesystem structure. 

Important metadata such as processor architecture, kernel version, filesystem type, and vendor-specific details are extracted and stored in the database. This information is essential for selecting appropriate emulation environments and vulnerability testing strategies.

\section{Firmware Emulation Implementation}
Firmware emulation is a core component of FIRMAI and is implemented using QEMU, a widely used open-source hardware emulator. Based on the detected architecture (ARM, MIPS, x86, etc.), the system automatically selects the appropriate QEMU emulator.

A customized Linux kernel is loaded along with the extracted firmware filesystem. Since many IoT firmware images use non-standard boot mechanisms, an emulation arbitration mechanism is implemented. This includes:
\begin{itemize}
	\item Modifying kernel boot parameters
	\item Emulating NVRAM using file-based storage
	\item Creating expected device nodes
	\item Dynamically configuring network interfaces
\end{itemize}

\section{Vulnerability Analysis Engine Implementation}
After successful firmware emulation, the vulnerability analysis engine performs automated security assessments. The engine first enumerates all exposed services and identifies their versions. These versions are correlated with known vulnerabilities using public CVE databases.

In addition to signature-based detection, dynamic testing techniques such as fuzzing are applied to web interfaces and network services. Input parameters are mutated to trigger unexpected behavior such as crashes, authentication bypasses, or command injection. All detected vulnerabilities are classified based on severity and documented with evidence. The system generates detailed vulnerability reports including CVE IDs, risk scores, affected services, and recommended mitigation steps.

\section{Network Scanning and Device Discovery Module}
The network scanning module identifies IoT devices connected to the target network. ARP-based scanning is used for fast device discovery at the data link layer, enabling detection even in environments with restrictive firewall rules.

Once devices are discovered, multi-threaded TCP scanning is performed to identify open ports and running services. Banner grabbing techniques are used to extract service information such as software names and versions.

Each discovered device is cataloged with details including IP address, MAC address, open ports, service banners, and response time. This information is later correlated with firmware vulnerabilities and malware detection results. 

\section{Device Identification and Fingerprinting}
Device identification is performed by analyzing MAC address OUIs, open port patterns, and service banners. The system maps MAC address prefixes to vendor information using standard vendor databases.

Service fingerprints and port combinations are matched against known IoT device signatures to determine device type and manufacturer. Confidence scores are assigned to each identification result, ensuring accuracy and transparency.

\section{Malware Detection and AI Model Implementation}
The malware detection module uses Artificial Intelligence to classify network traffic as benign or malicious. Network packets are captured using libpcap and processed into flow-level features such as packet size distribution, protocol usage, port frequency, and inter-arrival times.

A Convolutional Neural Network (CNN) is implemented using TensorFlow and Keras. The model is trained on labeled IoT malware datasets, enabling it to learn behavioral patterns of various malware families.

During real-time operation, captured traffic is passed through the trained model, which outputs classification results along with confidence scores. High-confidence malicious detections trigger alerts and are logged for further analysis.

\section{System Integration and Data Correlation}
All modules in FIRMAI are integrated using RESTful APIs and asynchronous message queues. Each module publishes its results to a central data repository.

A correlation engine links findings from different modules using common identifiers such as IP address, MAC address, and firmware hash. This enables the system to associate firmware vulnerabilities with live network behavior and detected malware activity.

The integrated view allows security analysts to understand attack paths and identify high-risk devices efficiently.

\section{Dashboard and User Interface Implementation}
The user interface is implemented as a web-based dashboard that provides real-time visibility into system operations. The dashboard displays:
\begin{itemize}
	\item Discovered IoT devices
	\item Firmware vulnerability details
	\item Malware detection alerts
	\item Risk scores and analysis reports
\end{itemize}

Role-based access control ensures that only authorized users can access sensitive information. Users can generate and download reports in multiple formats for documentation and compliance purposes.

\section{Deployment and Containerization}
The entire FIRMAI system is containerized using Docker, ensuring consistent deployment across different environments. Docker Compose is used to orchestrate multiple services such as backend APIs, databases, AI model servers, and the dashboard.

Containerization simplifies installation, scaling, and maintenance. The system can be deployed on local servers, institutional labs, or cloud platforms with minimal configuration changes.

\section{Security Considerations}
Security is enforced throughout the implementation. All communications between system components are secured using HTTPS. Sensitive data such as credentials and vulnerability information are encrypted at rest.

Authentication and authorization mechanisms prevent unauthorized access, and audit logs are maintained for all critical actions. Regular security testing ensures that the system itself does not introduce new vulnerabilities.