\chapter*{Abstract}
\addcontentsline{toc}{chapter}{Abstract}

The exponential growth of Internet of Things (IoT) devices has introduced unprecedented security challenges, particularly in firmware vulnerability detection and analysis. This research presents an innovative AI-powered IoT firmware vulnerability analyzer that integrates advanced emulation techniques, deep learning-based malware classification, and comprehensive network scanning capabilities. The system addresses critical gaps in current firmware security assessment methodologies by combining automated firmware emulation frameworks, transformer-based vulnerability detection models, and real-time network threat analysis. Through the integration of symbolic execution techniques, convolutional neural networks trained on IoT-23 datasets, and comprehensive vulnerability scanning protocols, the proposed framework achieves enhanced detection accuracy while maintaining operational efficiency. The system leverages modern AI techniques including transformer architectures for opcode sequence analysis, enabling identification of zero-day vulnerabilities and sophisticated attack patterns. Experimental validation demonstrates significant improvements in vulnerability detection rates compared to traditional static analysis tools, with particular effectiveness in identifying buffer overflow attacks, command injection vulnerabilities, and firmware-level malware infections. The framework provides automated threat response mechanisms, real-time monitoring capabilities, and comprehensive security assessment reports, making it suitable for both research applications and commercial IoT security implementations.

\vspace{1cm}
\noindent \textbf{Keywords:} Internet of Things (IoT), Firmware Emulation, Network Security, Deep Learning, Malware Classification, Convolutional Neural Networks (CNN), Vulnerability Analysis, Network Scanning, Cyber Security, Virtualization.
